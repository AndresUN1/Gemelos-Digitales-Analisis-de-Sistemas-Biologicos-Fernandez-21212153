
\documentclass[letterpaper,11pt]{article}
%%%%%%%%%%%%%%%%%%%%%%%%%%%%%%%%%%%%%%%%%%%%%%%%%%%%%%%%%%%%%%%%%%%%%%%%%%%%%%%%%%%%%%%%%%%%%%%%%%%%%%%%%%%%%%%%%%%%%%%%%%%%%%%%%%%%%%%%%%%%%%%%%%%%%%%%%%%%%%%%%%%%%%%%%%%%%%%%%%%%%%%%%%%%%%%%%%%%%%%%%%%%%%%%%%%%%%%%%%%%%%%%%%%%%%%%%%%%%%%%%%%%%%%%%%%%
\usepackage{graphicx}
\usepackage{amsmath,amsfonts,amssymb,amsthm,float}
\usepackage{hyperref}
\usepackage[utf8]{inputenc}
\usepackage[left=2cm, right=2cm, top=2cm, bottom=2cm]{geometry}

\setcounter{MaxMatrixCols}{10}
%TCIDATA{OutputFilter=LATEX.DLL}
%TCIDATA{Version=5.50.0.2953}
%TCIDATA{<META NAME="SaveForMode" CONTENT="1">}
%TCIDATA{BibliographyScheme=BibTeX}
%TCIDATA{LastRevised=Monday, May 12, 2025 15:38:04}
%TCIDATA{<META NAME="GraphicsSave" CONTENT="32">}
%TCIDATA{ComputeDefs=
%$R$
%}


\input{tcilatex}
\renewcommand{\baselinestretch}{1.15}
\setlength{\parindent}{0pt}
\setlength{\parskip}{0.5\baselineskip}
\pretolerance=2000 \tolerance=3000
\renewcommand{\abstractname}{Resumen}

\begin{document}

\title{Analisis de sistemas biologicos}
\author{Fernandez Esquivel Hector Andres $\left[ 21212153\right] $ \\
%EndAName
Departamento de Ingenier\'{\i}a El\'{e}ctrica y Electr\'{o}nica\\
Tecnol\'{o}gico Nacional de M\'{e}xico / Instituto Tecnol\'{o}gico de Tijuana%
}
\maketitle

\noindent \textbf{Palabras clave: }Palabra clave 1; Palabra clave 2; Palabra
clave 3; Palabra clave 4; Palabra clave 5.

\noindent Correo: \textbf{l21212153@tectijuana.edu.mx}

\noindent \noindent Carrera: \textbf{Ingenier\'{\i}a Biom\'{e}dica }

\noindent Asignatura: \textbf{Gemelos Digitales}

\noindent Profesor: \href{https://biomath.xyz/}{\textbf{Dr. Paul Antonio
Valle Trujillo}} (paul.valle@tectijuana.edu.mx)

\section{Modelo matematico}

\bigskip El modelo matematico se compone por las siguientes tres Ecuaciones
Dgerenciales Orfinarias (EDOs)\ de primer orden:

\begin{eqnarray}
\dot{x} &=&r_{1}x(1-b_{1}x)-a_{12}xy-a_{13}xz,  \label{dx} \\
\dot{y} &=&r_{2}y(1-b_{2}y)-a_{21}xy,  \label{dy} \\
\dot{z} &=&(r_{3}-a_{31})xz-d_{3}z+\rho _{i},  \label{dz}
\end{eqnarray}

donde $x(t)$ es la poblacion de celulas anormales, $y(t)$ la poblacion de
celulas normales y $z(t)$ la poblacion de celulas efectoras, ademas el
tiempo $t$ se mide en dias.

Comentarios sobre el modelo:

\begin{itemize}
\item 
\begin{enumerate}
\item El crecimiento de las poblaciones de celulas anormales y normales se
describe mediante la ley de crecimiento logistico.

\item Las celulas normales y efectoras afectan a la poblacion de celulas
anormales mediante la ley de accion de masas.

\item La poblacion de celulas efectoras, solamente afecta a la poblacion de
celulas anormales.

\item Las celulas anormales disminuyen el crecimiento de las celulas
normales.

\item El crecimiento o decrecimiento de las celulas efectoras depende de los
valores en las tasas de reclutamiento e inactivacion ocasionado por la
interaccion con celulas anormales.

\item Las celulas efectoras tienen una tasa de muerte constante dentro del
sistema.

\item Las celulas efectoras se pueden potenciar de forma externa mediante el
parametro de tratamiento.

\item La dinamica del sistema es de la forma presa-depredador de
Lotka-Volterra.

\item Debido a que el sistema describe la concentracion de poblaciones
celulares con respecto al tiempo, sus soluciones deben ser no negativas para
condiciones iniciales no negativas, de lo contrario, se perderia el
significado biologico del sistema.
\end{enumerate}
\end{itemize}

\section{Analisis de positividad}

En esta seccion se aplica el lema de positividad para sistemas dinamicos no
lineales, por lo que se realizan las siguientes evaluaciones:

\begin{equation*}
\left\vert \dot{x}\right\vert _{x=0}=r_{1}(0)(1-b_{1}\left( 0\right)
)-a_{12}\left( 0\right) y-a_{13}\left( 0\right) z=0
\end{equation*}

\begin{equation*}
\left\vert \dot{y}\right\vert _{y=0}=r_{2}\left( 0\right) (1-b_{2}\left(
0\right) )-a_{21}x\left( 0\right) =0
\end{equation*}

\begin{equation*}
\left\vert \dot{z}\right\vert _{z=0}=(r_{3}-a_{31})x\left( 0\right)
-d_{3}\left( 0\right) +\rho _{i}=\rho _{i}
\end{equation*}

Por lo tanto, de acuerdo con De Leenher \&\ Aeyels [1], se concluye el
siguiente resultado:

\bigskip

\textbf{Resultado I. Positividad}:\ Las soluciones $[x(t),y\left( t\right)
,z\left( t\right) ]$ y semi-trayectorias positivas $(\Gamma ^{+})$ del
sistema $\left( \ref{dx}\right) -\left( \ref{dz}\right) $ seran
positivamente invariantes y para cada condicion inicial no negativa $%
[x\left( 0\right) ,y\left( 0\right) ,z\left( 0\right) \geq 0]$ se
localizaran en el siguiente dominio:

\begin{equation*}
R_{+,0}^{3}=\left\{ x\left( t\right) ,y\left( t\right) ,z\left( t\right)
\geq 0\right\}
\end{equation*}

Referencia:

\begin{enumerate}
\item \lbrack 1]P. De Leenheer \& D. Aeyels, \textquotedblleft Stability
properties of equilibria of classes of cooperative
systems,\textquotedblright\ IEEE Transactions on Automatic Control, vol. 46,
no. 12, pp. 1996--2001, 2001, doi: https://doi.org/10.1109/9.975508.
\end{enumerate}

\section{Localizacion de conjuntos compactos invariantes}

Perimero se debe proponer una funcion localizadora, para sistemas biologicos
con dinamica localizada en el ortante no negativo, se sugiere explorar las
siguientes funciones:

\begin{eqnarray*}
h_{1} &=&x, \\
h_{2} &=&y, \\
h_{3} &=&z, \\
h_{4} &=&x+y+z, \\
h_{5} &=&x+z, \\
h_{6} &=&x+y, \\
h_{7} &=&y+z.
\end{eqnarray*}

Nota: Con base en la estructura del sistema, se observa que las variables $%
x(t)$ y $y\left( t\right) $, tienen los siguientes limites inferiores y
superiores:

\begin{eqnarray*}
0 &\leq &x\left( t\right) \leq 1 \\
0 &\leq &y\left( t\right) \leq 1
\end{eqnarray*}%
esto corresponde con la ley de crecimiento logistico (crecimiento de tipo
sigmoidal), que tiende a cero al menos infinito y a uno hacia el infinito.

Se explora la siguiente funcion localizadora:

\begin{equation*}
h_{1}=x,
\end{equation*}%
y se calcula su derivada de Lie (derivada temporal o derivada implicita con
respecto al tiempo)

\begin{equation*}
L_{f}h_{1}=\frac{dx}{dt}=\dot{x}=r_{1}x(1-b_{1}x)-a_{12}xy-a_{13}xz,
\end{equation*}%
con lo cual, se formula el conjunto $S\left( h_{1}\right) =\left\{
L_{f}h_{1}=0\right\} $, es decir,

\begin{equation*}
S(h_{1})=\left\{ r_{1}x(1-b_{1}x)-a_{12}xy-a_{13}xz=0\right\} ,
\end{equation*}
$\allowbreak $se observa que este conjunto puede reescribirse de la
siguiente forma:

\begin{equation*}
S(h_{1})=\left\{ r_{1}-r_{1}b_{1}x-a_{12}y-a_{13}z=0\right\} \cup \left\{
x=0\right\} ,
\end{equation*}%
ahora, se reescribe la primera parte del conjunto, despejando la variable de
interes:

\begin{equation*}
S(h_{1})=\left\{ x=\frac{1}{b_{1}}-\frac{a_{12}}{r_{1}b_{1}}y-\frac{a_{13}}{%
r_{1}b_{1}}z=0\right\} \cup \left\{ x=0\right\} ,
\end{equation*}%
con base en lo anterior se concluye lo siguiente:

\begin{equation*}
K\left( h_{1}\right) =\left\{ x_{\inf }=0\leq x\left( t\right) \leq x_{\max
}=\frac{1}{b_{1}}\right\} ,
\end{equation*}%
es decir, el valor minimo que puede tener la solucion $x\left( t\right) $ es
de cero, mientras que, el valor maximo que puede alcanzar esta solucion
cuando $y=z=0$, es de uno (recordando que el sistema esta normalizado).

\bigskip 

Ahora, se explora la siguiente funcion localizadora:

\begin{equation*}
h_{2}=y
\end{equation*}%
y se calcula su derivada de Lie:

\begin{equation*}
L_{f}h_{2}=r_{2}y(1-b_{2}y)-a_{21}xy,
\end{equation*}%
entonces, el conjunto $S\left( h_{2}\right) =\left\{ L_{f}h_{2}=0\right\} $,
esta dado por lo siguiente:

\begin{equation*}
S(h_{2})=\left\{ y=\frac{1}{b_{2}}-\frac{a_{21}}{r_{2}b_{2}}x\right\} \cup
\left\{ y=0\right\} ,
\end{equation*}%
con base en lo anterior, se concluye el siguiente resultado:

\begin{equation*}
K\left( h_{2}\right) =\left\{ y_{\inf }=0\leq y\left( t\right) \leq y_{\max
}=\frac{1}{b_{2}}\right\} ,
\end{equation*}

Ahora, con base en la siguiente funcion localizadora:

\begin{equation*}
h_{3}=z
\end{equation*}%
al calcular su derivada de Lie:

\begin{equation*}
L_{f}h_{3}=(r_{3}-a_{31})xz-d_{3}z+\rho _{i}
\end{equation*}%
se obtiene el conjunto $S\left( h_{3}\right) $ como se muestra a
continuacion:

\begin{equation*}
S\left( h_{3}\right) =\left\{ L_{f}h_{3}=0\right\} =\left\{ \left(
r_{3}-a_{31}\right) xz-d_{3}z+\rho _{i}=0\right\} ,
\end{equation*}%
donde, al observar los valores de los parametros, se construye la siguiente
condicion:

\begin{equation*}
r_{3}>a_{31},
\end{equation*}%
por lo tanto, se reescribe el conjunto $S\left( h_{3}\right) $ de la
siguiente forma:

\begin{equation*}
S\left( h_{3}\right) =\left\{ z=\frac{\rho _{i}}{d_{3}}+\frac{r_{3}-a_{31}}{%
d_{3}}xz\right\} ,
\end{equation*}%
por lo tanto, se observa que, la solucion tiene el siguiente limite inferior:

\begin{equation*}
K\left( z\right) =\left\{ z\left( t\right) \geq \frac{\rho _{i}}{d_{3}}%
\right\} 
\end{equation*}%
recordando que $\rho _{i}$\bigskip\ es el parametro de tratamiento/terapia
(o parametro de control), que puede tener valores no negativos, es decir, $%
\rho _{i}\geq 0$.

Por lo tanto, con base en el resultado anterior, se procede a aplicar el
denominado Teorema Iterativo del metodo de LCCI, entonce, se reescribe el
conjunto $S\left( h_{1}\right) $ como se muestra a continuacion:

\begin{eqnarray*}
S(h_{1}) &=&\left\{ r_{1}-r_{1}b_{1}x-a_{12}y-a_{13}z=0\right\} \cup \left\{
x=0\right\} , \\
S(h_{1})\cap K\left( z\right)  &\subset &\left\{ x=\frac{1}{b_{1}}-\frac{%
a_{12}}{r_{1}b_{1}}y-\frac{a_{13}}{r_{1}b_{1}}z_{\inf }\right\} ,
\end{eqnarray*}%
ahora, al descartar el termino negativo de $y$, se concluye el siguiente
limite superior para la variable $x\left( t\right) $:

\begin{equation*}
K_{x}=\left\{ x_{\inf }=0\leq x\left( t\right) \leq x_{\sup }=\frac{1}{b_{1}}%
-\frac{a_{13}}{r_{1}b_{1}d_{3}}\rho _{i}\right\} ,
\end{equation*}

Finalmente, se toma la siguiente funcion localizadora:

\begin{equation*}
h_{4}=\alpha x+z
\end{equation*}%
cuya derivada de Lie se muestra a continuacion:

\begin{equation*}
L_{f}h_{4}=a\left[ r_{1}x(1-b_{1}x)-a_{12}xy-a_{13}xz\right] +\left(
r_{3}-a_{31}\right) xz-d_{3}z+\rho _{i}
\end{equation*}%
y se determina el conjunto $S\left( h_{4}\right) =\left\{
L_{f}h_{4}=0\right\} \,$\ de la siguiente forma:

\begin{eqnarray*}
S\left( h_{4}\right)  &=&\left\{ \alpha r_{1}x-b_{1}\alpha r_{1}x^{2}-\alpha
a_{12}xy-\alpha a_{13}xz+\left( r_{3}-a_{31}\right) xz-d_{3}z+\rho
_{i}=0\right\} , \\
S\left( h_{4}\right)  &=&\left\{ \rho _{i}-b_{1}\alpha r_{1}x^{2}+\alpha
r_{1}x-\alpha a_{12}xy-\left( \alpha a_{13}-r_{3}+a_{31}\right)
xz-d_{3}z=0\right\} ,
\end{eqnarray*}%
para asegurar que todos los teminos cruzados/no lineales/cuadraticos, sean
negativos, se impone la siguiente condicion:

\begin{eqnarray*}
\alpha a_{13}-r_{3}+a_{31} &>&0, \\
\alpha  &>&\frac{r_{3}-a_{31}}{a_{13}},
\end{eqnarray*}%
ahora, la funcion localizadora se puede expresar de esta forma:

\begin{equation*}
z=h_{4}-\alpha x,
\end{equation*}%
para sustituir en la siguiente expresion:

\begin{equation*}
S\left( h_{4}\right) =\left\{ d_{3}z=\rho _{i}-b_{1}\alpha r_{1}x^{2}+\alpha
r_{1}x-\alpha a_{12}xy-\left( \alpha a_{13}-r_{3}+a_{31}\right) xz\right\} ,
\end{equation*}%
es decir,

\begin{eqnarray*}
S\left( h_{4}\right)  &=&\left\{ d_{3}\left( h_{4}-\alpha x\right) =\rho
_{i}-b_{1}\alpha r_{1}x^{2}+\alpha r_{1}x-\alpha a_{12}xy-\left( \alpha
a_{13}-r_{3}+a_{31}\right) xz\right\} , \\
S\left( h_{4}\right)  &=&\left\{ d_{3}h_{4}=\rho _{i}-b_{1}\alpha
r_{1}x^{2}+\left( \alpha r_{1}+d_{3}\alpha \right) x-\alpha a_{12}xy-\left(
\alpha a_{13}-r_{3}+a_{31}\right) xz\right\} , \\
S\left( h_{4}\right)  &=&\left\{ h_{4}=\frac{\rho _{i}}{d_{3}}-\frac{%
b_{1}\alpha r_{1}}{d_{3}}x^{2}+\frac{\alpha r_{1}+d_{3}\alpha }{d_{3}}x-%
\frac{\alpha a_{12}}{d_{3}}xy-\frac{\alpha a_{13}-r_{3}+a_{31}}{d_{3}}%
xz\right\} ,
\end{eqnarray*}%
para continuar con el proceso, primero se debe completar el cuadrado con los
siguientes dos terminos:

\begin{equation*}
-\frac{b_{1}\alpha r_{1}}{d_{3}}x^{2}+\frac{\alpha r_{1}+d_{3}\alpha }{d_{3}}%
x=-Ax^{2}+Bx=-A\left( x-\frac{B}{2A}\right) ^{2}+\frac{B^{2}}{4A}
\end{equation*}%
y se sustituye en el conjunto $S\left( h_{4}\right) $

\begin{equation*}
S\left( h_{4}\right) =\left\{ h_{4}=\frac{\rho _{i}}{d_{3}}+\frac{B^{2}}{4A}%
-A\left( x-\frac{B}{2A}\right) ^{2}-\frac{\alpha a_{12}}{d_{3}}xy-\frac{%
\alpha a_{13}-r_{3}+a_{31}}{d_{3}}xz\right\} ,
\end{equation*}%
por lo tanto, se concluye el siguiente limite superior para la funcion $h_{4}
$:

\begin{equation*}
K\left( h_{4}\right) =\left\{ ax\left( t\right) +z\left( t\right) \leq \frac{%
\rho _{i}}{d_{3}}+\frac{\alpha \left( d_{3}+r_{1}\right) ^{2}}{%
4b_{1}d_{3}r_{1}}\right\} ,
\end{equation*}%
y se aproxima el siguiente limite superior para la variable $z\left(
t\right) $:

\begin{equation*}
K_{z}=\left\{ z_{\inf }=\frac{\rho _{i}}{d_{3}}\leq z\left( t\right) \leq
z_{\sup }=\frac{\rho _{i}}{d_{3}}+\frac{\alpha \left( d_{3}+r_{1}\right) ^{2}%
}{4b_{1}d_{3}r_{1}}\right\} .
\end{equation*}

Con base en lo mostrado en esta seccion, se concluye el siguiente resultado:

\bigskip 

\textbf{Resultado II: Dominio de localizacion}\textit{: Todos los conjuntos
compactos invariantes del sistema }$\left( \ref{dx}\right) -\left( \ref{dz}%
\right) $\textit{\ se encuentran localizados dentro o en las fronteras del
siguiente dominio de localizacion:}

\bigskip 
\begin{equation*}
K_{xyz}=K_{x}\cap K_{y}\cap K_{z},
\end{equation*}%
\textit{donde}

\bigskip 
\begin{eqnarray*}
K_{x} &=&\left\{ x_{\inf }=0\leq x\left( t\right) \leq x_{\sup }=\frac{1}{%
b_{1}}-\frac{a_{13}}{r_{1}b_{1}d_{3}}\rho _{i}\right\} , \\
K_{y} &=&\left\{ y_{\inf }=0\leq y\left( t\right) \leq y_{\sup }=\frac{1}{%
b_{2}}\right\} , \\
K_{z} &=&\left\{ z_{\inf }=\frac{\rho _{i}}{d_{3}}\leq z\left( t\right) \leq
z_{\sup }=\frac{\rho _{i}}{d_{3}}+\frac{\alpha \left( d_{3}+r_{1}\right) ^{2}%
}{4b_{1}d_{3}r_{1}}\right\} .
\end{eqnarray*}

\subsection{\protect\bigskip No Existencia de conjuntos compactos invariantes%
}

A partir del resultado mostrado en el conjunto $K_{x}$, es posible
establecer lo siguiente con respecto a la existencia de conjuntos compactos
invariantes para la variable $x\left( t\right) $:

\bigskip 

\textbf{Resultado III: No existencia}\textit{: Si la siguiente condicion
sobre el parametro de tratamiento/terapia se cumple:}

\begin{equation*}
\frac{1}{b_{1}}-\frac{a_{13}}{r_{1}b_{1}d_{3}}\rho _{i}\leq 0,
\end{equation*}%
\textit{es decir,}

\bigskip 
\begin{equation*}
\rho _{i}\geq \frac{r_{1}d_{3}}{a_{13}},
\end{equation*}%
\textit{entonces, se puede asegurar la no existencia de conjuntos compactos
invariantes fuera del plano }$x=0$\textit{, por lo tanto, cualquier dinamica
que pueda exhibir el sistema, estara localizada dentro o en las fronteras
del siguiente dominio:}

\begin{equation*}
K_{xyz}=\left\{ x=0\right\} \cap K_{y}\cap K_{z},
\end{equation*}

\subsection{Puntos de equilibrio}

Para calcular los puntos de equilibrio del sistema $\left( \ref{dx}\right)
-\left( \ref{dz}\right) $, se igualan a cero las ecuaciones como se muestra
a continuacion:

$\func{assume}\left( r_{1},\func{positive}\right) =\allowbreak \left(
0,\infty \right) $

$\func{assume}\left( b_{1},\func{positive}\right) =\allowbreak \left(
0,\infty \right) $

$\func{assume}\left( a_{12},\func{positive}\right) =\allowbreak \left(
0,\infty \right) $

$\func{assume}\left( a_{13},\func{positive}\right) =\allowbreak \left(
0,\infty \right) $

$\func{assume}\left( r_{2},\func{positive}\right) =\allowbreak \left(
0,\infty \right) $

$\func{assume}\left( a_{21},\func{positive}\right) =\allowbreak \left(
0,\infty \right) $

$\func{assume}\left( r_{3},\func{positive}\right) =\allowbreak \left(
0,\infty \right) $

$\func{assume}\left( a_{31},\func{positive}\right) =\allowbreak \left(
0,\infty \right) $

$\func{assume}\left( d_{3},\func{positive}\right) =\allowbreak \left(
0,\infty \right) $

\bigskip Primero, se calculan los equilibrios asumiendo $\rho _{i}=0:$

\begin{eqnarray*}
0 &=&r_{1}x(1-b_{1}x)-a_{12}xy-a_{13}xz \\
0 &=&r_{2}y(1-b_{2}y)-a_{21}xy \\
0 &=&(r_{3}-a_{31})xz-d_{3}z
\end{eqnarray*}

\begin{eqnarray*}
&&\left[ x=0,y=0,z=0\right]  \\
&&\left[ x=\frac{d_{3}}{r_{3}-a_{31}},y=0,z=-\frac{1}{%
r_{3}a_{13}-a_{13}a_{31}}\left(
-r_{1}r_{3}+r_{1}a_{31}+b_{1}d_{3}r_{1}\right) \right]  \\
&&\left[ x=\frac{d_{3}}{r_{3}-a_{31}},y=-\frac{1}{%
b_{2}r_{2}r_{3}-b_{2}r_{2}a_{31}}\left(
d_{3}a_{21}-r_{2}r_{3}+r_{2}a_{31}\right) ,z=\frac{1}{%
b_{2}r_{2}r_{3}a_{13}-b_{2}r_{2}a_{13}a_{31}}\left(
d_{3}a_{12}a_{21}-r_{2}r_{3}a_{12}+r_{2}a_{12}a_{31}+b_{2}r_{1}r_{2}r_{3}-b_{2}r_{1}r_{2}a_{31}-b_{1}b_{2}d_{3}r_{1}r_{2}\right) %
\right]  \\
&&\left[ x=\frac{1}{a_{21}}\left( r_{2}-b_{2}r_{2}\frac{%
r_{1}a_{21}-b_{1}r_{1}r_{2}}{a_{12}a_{21}-b_{1}b_{2}r_{1}r_{2}}\right) ,y=%
\frac{r_{1}a_{21}-b_{1}r_{1}r_{2}}{a_{12}a_{21}-b_{1}b_{2}r_{1}r_{2}},z=0%
\right]  \\
&&\left[ x=0,y=\frac{1}{b_{2}},z=0\right] 
\end{eqnarray*}

\bigskip $\func{assume}\left( \rho _{i},\func{positive}\right) =\allowbreak
\left( 0,\infty \right) $

Ahora, considerando  $\rho _{i}>0:$

\begin{eqnarray*}
0 &=&r_{1}x(1-b_{1}x)-a_{12}xy-a_{13}xz \\
0 &=&r_{2}y(1-b_{2}y)-a_{21}xy \\
0 &=&(r_{3}-a_{31})xz-d_{3}z
\end{eqnarray*}

Esto se realizara en MATLAB.\bigskip 

\subsection{\protect\bigskip Condiciones de eliminacion}

Las condiciones de eliminacion se establecen sobre el parametro de
tratamiento/terapia o control, y se determinan al aplicar la teoria de
estabilidad en el sentido de Lyapunov, particularmente el metodo directo de
Lyapunov.

Se propone la siguiente funcion candidata de Lyapunov:

\begin{equation*}
V=x,
\end{equation*}%
y se calcula su derivada

\begin{equation*}
\dot{V}=\dot{x}=r_{1}x(1-b_{1}x)-a_{12}xy-a_{13}xz,
\end{equation*}%
se reescribe la derivada de la siguiente forma:%
\begin{equation*}
\dot{V}=(r_{1}-r_{1}b_{1}x-a_{12}y-a_{13}z),
\end{equation*}%
ahora, al considerar los resultados del dominio de localizacion y evaluar la
derivada en este, es decir,%
\begin{equation*}
\left\vert \dot{V}\right\vert _{K_{xyz}},
\end{equation*}%
se tiene lo siguiente:%
\begin{equation*}
\dot{V}=\left( r_{1}-a13z_{\inf }\right) x\leq 0,
\end{equation*}%
a partir de esta expresion, se establece la siguiente condicion:%
\begin{equation*}
r_{1}-a_{13}\frac{\rho _{i}}{d_{3}}<0,
\end{equation*}%
por lo tanto. se despeja el parametro de tratamiento/terapia o control%
\begin{equation*}
\rho _{i}>\frac{d_{3}r_{1}}{a_{13}},
\end{equation*}%
y se establece el siguiente resultado:

\bigskip 

\textbf{Resultado IV:\ Condiciones de eliminacion}: \textit{Si la siguiente
condicion se cumple:}%
\begin{equation*}
\rho _{i}>\frac{d_{3}r_{1}}{a_{13}},
\end{equation*}

\textit{entonces, se puede asegurar la eliminacion de la poblacion descrita
por la variable }$x\left( t\right) $\textit{, es decir,}%
\begin{equation*}
\lim_{t\rightarrow \infty }x\left( t\right) =0.
\end{equation*}

\textit{\bigskip }

\bigskip 

\bigskip 

\bigskip 

\bigskip 

\bigskip 

\bigskip 

\bigskip 

\bigskip 

\bigskip 

\bigskip 

\bigskip 

\bigskip 

\bigskip 

\bigskip 

\begin{eqnarray*}
\dot{x} &=&r_{1}x(1-b_{1}x)-a_{12}xy-a_{13}xz, \\
\dot{y} &=&r_{2}y(1-b_{2}y)-a_{21}xy, \\
\dot{z} &=&(r_{3}-a_{31})xz-d_{3}z+\rho _{i},
\end{eqnarray*}

\end{document}
